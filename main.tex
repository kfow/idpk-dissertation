\documentclass{l4proj}
%% Language and font encodings
\usepackage[english]{babel}
\usepackage[utf8x]{inputenc}
\usepackage[T1]{fontenc}

%% Sets page size and margins
\usepackage[a4paper,top=3cm,bottom=2cm,left=3cm,right=3cm,marginparwidth=1.75cm]{geometry}
\usepackage[colorinlistoftodos]{todonotes}
\usepackage[colorlinks=true, allcolors=blue]{hyperref}
\usepackage{listings}
\usepackage{minted}
\usemintedstyle{emacs}
\usepackage{pgfplots}
\pgfplotsset{compat=1.8}
\usepackage{filecontents}
\usepackage{graphicx}
\usepackage{adjustbox}
\usepackage{tikz}
\usepackage{enumerate}
\usepackage{enumitem}
\usepackage{url}
\usepackage{amssymb}
\usepackage{amsmath}
\newcommand{\code}[1]{\texttt{#1}}

\title{Public vs. Private: \newline Identifying Public Domain Knowledge}
\author{Kelvin Fowler}
\date{\today}

\begin{document}
\maketitle

\begin{abstract}
The current system of sensitivity review used in the archival process of government documents is slow and insecure. Information Retrieval techniques can be leveraged to vastly improve this operation. This project aims to provide a system and an interface, incorporating these information retrieval techniques, to allow archivists to identify information from documents which already exists within the public domain.
Expand this.
\end{abstract}

% \renewcommand{\abstractname}{Acknowledgements}
% \begin{abstract}
% I would like to thank my supervisors Dr Craig Macdonald and Graham McDonald for their continued assistance throughout the course of this project.
% \end{abstract}

\educationalconsent

\tableofcontents

%%%%%%%%%%%%%%%%%%%%%%%%%%%%%%%%%%%%%%%%%%%%%%
\chapter{Introduction}
\pagenumbering{arabic}

\section{Aims}
The aims of this project.
Help government archivists to identify when a document contains information already in the public domain. Armed with this knowledge, the archivist can make a more informed decision regarding the public availability of a document.

\section{Motivations}
The current system of document review involves no official public domain knowledge checking process. Often, archivists are forced to use regular web search engines, which poses a potential security risk as certain queries regarding document contents are, by their very nature, security sensitive.

\section{Structure}
This paper will discuss the challenges and decisions involved in the design, implementation and evaluation of this system. We will identify the project's relevance through reviewing some related literature. We will take an in-depth look at the planning and design of the system before moving on to explain the key implementation challenges and solutions.
We will finally discuss the evaluation techniques used to improve the system and conclude with learnings and conclusions.

\chapter{Related Literature}
\section{Sensitivity Review}
\paragraph{Towards a Classifier for Digital Sensitivity Review}
This paper discusses other methods to help alleviate concerns related to sensitivity review as more and more records move from paper to digital \cite{mcdonald2014towards}. It identifies that named entities and document sentiment may be used to automatically identify potential sensitivities. Although related to this project, this work aims to identify sensitive documents before review, rather than to help facilitate review by identifying public domain knowledge.

\paragraph{On Using Information Retrieval for the Selection and Sensitivity Review of Digital Public Records}
This is another paper which discusses the challenges and potential avenues for improving the digital sensitivity review process\cite{gollins2014using}. A key, relevant point being that reviewers are reluctant to trust technology alone in this type of work. As such, technology which can assist the manual review process (such as that developed in this project) will be increasingly important as the review process is improved and iterated upon.

\subsection{From Paper to Digital Processing}
The current sensitivity review procedures used by government archivists are less then ideal. Paper documents must be read in full and any issues must be taken up with other departments. Moving to digital records increases greatly the number of documents to be reviewed and, as such, this process is unreasonable. These concerns were key in the development of the above research projects, as well as this one. Records Review \cite{allan2014record} outlines these concerns.

\subsection{Assistive Technology for Sensitivity}

\section{Information Retrieval}
\subsection{Search Engines}
Terrier, Lucene
\subsection{Ad-hoc Retrieval}

\subsection{High Recall Tasks}
High recall task focus on retrieving as many relevant results as possible, rather than high precision task which focus on retrieval of results which are most relevant.
One such field where this type of retrieval is particularly sought after is E-Discovery, that is, the discovery of all relevant documents pertaining the opposing party in civil litigation. \\ \\ Patent Search

Oard, D. W., \& Webber, W. Information retrieval for e-discovery. Foundations and Trends® in Information Retrieval, 7(2–3), 99-237.(2013)
Provides a good introduction to e-discovery
\subsection{Entities}
Wikify!, Other papers

\section{Gap/Need}
M. Moss. Where Have All the Files Gone? Lost in Action Points Every One? J. Contemporary History, 47(4), 2012.
See P867 for some background on the introduction of born digital documents in government and P872 for a bit about 'the gap’ you are addressing.

%%%%%%%%%%%%%%%%%%%%%%%%%%%%%%%%%%%%%%%%%%%%%%%%
\chapter{Requirements and Design}
\section{Requirements}
Throughout the project, requirements were identified and categorised using the MoSCoW method. This means that each requirement is identified as either Must Have, Should Have, Could Have or Would Like to Have.
Initially these requirements were elicited from the project supervisors, who had extensive knowledge of the problem domain, through related research and a working relationship with the project stakeholders, archivists in the British Government(?).
These initial requirements allowed the developer to begin work on the project, after which more detailed requirements were identified through discussion and experimentation.

\subsection{Functional Requirements}
The functional requirements of the project were further ratified through user stories. 
TODO: Reference for Virtues of user stories + reference?
\paragraph{Must Have\\}
\begin{enumerate}[label=\textbf{M.\arabic*}]
\item Allow user to identify when a document contains public domain knowledge. \\
\textit{As a} document reviewer, \\
\textit{I want} the ability to decide that a document contains public domain knowledge \\
\textit{So that} I can make a decision regarding it's status. \\

\item Front end user interface displaying source document for review and target documents automatically identified by the software. \\
\textit{As a} document reviewer, \\
\textit{I want} to view both the document up for review and associated press and Wiki articles \\
\textit{So that} I can easily compare contents of both and see what is known in the public domain. \\
\item Trials of different retrieval methods and some form of quantitative analysis based on these. \\
\textit{As a} document reviewer, \\
\textit{I want} to be confident that this software will return the most relevant results in reasonable time, \\
\textit{So that} I can trust the service to return excellent related documents.
\end{enumerate}

\paragraph{Should Have}
\begin{enumerate}[label=\textbf{S.\arabic*}]
\paragraph{Should Have}
\item Generate suggested queries to allow user to refine returned related documents. \\
\textit{As a} document reviewer, \\
\textit{I want} to refine search terms for related documents, \\
\textit{So that} more relevant documents are returned.

\item Ability to run custom queries. \\
\textit{As a} document reviewer, \\
\textit{I want} to write my own queries, \\
\textit{So that} I can retrieve the best results and be sure that the automatic query system is performing as well as possible.

\item Date range searches. \\
\textit{As a} document reviewer, \\
\textit{I want} to limit date range for retrieval of related documents, \\
\textit{So that} I can ensure relevance of related documents.

\paragraph{Could Have}
\end{enumerate}
\begin{enumerate}[label=\textbf{C.\arabic*}]
\item Machine learning to improve system performance over time based on usage by reviewers. \\
\textit{As a} document reviewer, \\
\textit{I want} my actions using the system to improve it’s performance to better suit my needs, \\
\textit{So that} it is easier to find related documents that I would deem relevant.
\item Tf-idf (Term Frequency - Inverse Document Frequency) analysis to better identify relevant documents. \\
\textit{As a} document reviewer, \\
\textit{I want} to only have returned to me related documents which have been identified as containing search terms in meaningful ways, \\
\textit{So that} I can be sure the returned documents contain information as relevant as possible to the search terms.
\end{enumerate}

\paragraph{Would Like To Have}
These are requirements which are outside the scope of this project and are not deemed important enough to be implemented.
\begin{enumerate}[label=\textbf{W.\arabic*}]
\item Wikify! Like functionality to automatically highlight and link to concepts in source documents which have wikipedia articles associated with them. \\
\textit{As a} document reviewer, \\
\textit{I want} to have easy access to wikipedia articles of items mentioned in the document I am reviewing, \\
\textit{So that} I can easily find out more about concepts mentioned in documents.
\end{enumerate}

\subsection{Non-Functional Requirements}
\begin{enumerate}[label=\textbf{NF.\arabic*}]
\item Easy to use user interface.
\item Results must be relevant.
\item Must return results in reasonable time.
\item Scalability to handle new large amounts of files.
\item Present results in an intuitive and understandable way.
\end{enumerate}

\section{Design and Architecture}
The application is based on a client-server design. More specifically, a web application acts as a client which interfaces with a RESTful API acting as a server.
The client communicates with the server using HTTP requests which request information from the server or update information on the server.
The web application maintains no state between usages, but rather requests all necessary data from the server at upon loading. Having no state system in the client application greatly reduces the complexity of the web application.
State and persistence is managed on the server entirely using Terrier and Trec files. This will be discussed in more detail in the Server Architecture section.

\subsection{Server Architecture}
\paragraph{Information Retrieval}
For the information retrieval needs of the project, Terrier was used. Terrier is a open-source search engine developed at The University of Glasgow, with an extensive Java API allowing it to be easily used for development of various IR tasks \cite{terrier} \cite{macdonald2012puppy}.
Terrier, being developed at The University of Glasgow, was the best and obvious choice of IR technology for the project. The local expertise meant that problems encountered could be easily resolved or explained, ensuring minimal downtime due to learning.
Other potential choices for this section could have been Apache Lucene or MG4J as recommended in Middleton-Baeza's Comparison of Open Source Search Engines \cite{middleton2007comparison}.

\paragraph{Persistence}
During indexing, one can save metadata to be retrieved for each document in the index. This provides an excellent opportunity to save some post-analysis information within the index data-structure.
The documents pertinent to the project (both source and target) are stored within the server file system as .trec files. There is one document per file.
Importantly, during indexing, Terrier uses a provided \code{collection.spec} file in order to identify files to index. It then assigns these documents ids in order of analysis. This order corresponds to the line number of the \code{collection.spec}. Leveraging this fact, we can find the line number corresponding to the file path of a document by reading the line of the \code{collection.spec} identified by the document id.
Furthermore, we can write to these \code{.trec} files, and save within them, information too large to save in the meta-index generated during terrier indexing.
These techniques produce a fully functional state system capable of retaining all the necessary information needed for the system.
This design entirely eliminates the need for a DBMS (Database Management System) like PostgreSQL and the complicated Java Object to Database Model mapping that often comes with such designs.
We leverage the capabilities of already present technologies to provide persistent state without wasting extensive resources.

\paragraph{Natural Language Processing}
Stanford Natural Language Processing was used as an NLP framework to allow for greater accuracy in queries \cite{manning-EtAl:2014:P14-5}. Provided is a Java API which allows text to be analysed and annotated. More specifically, the named entity identification capability of StandordNLP was used \cite{finkel2005incorporating}.
Stanford NLP was initially chosen due to its preferable documentation. It also offers the ability to identify specifically which tokenisers to use when processing some text allowing the developer to streamline the NLP process to their specific use-case.

\paragraph{RESTful API}
The entire back-end application is written in Java. This was an obvious choice due to Terriers Java API and the necessity to build upon this to create a viable product.
Jersey is a framework which provides a reference implementation of the JAX-RS API as defined by Oracle \cite{jersey} \cite{jaxrsapi}.
Jackson is used alongside Jersey to provide support for JSON \cite{jackson}. This allows the frontend and backend to communicate using one format. It handles the conversion of JSON to Java Objects and vice versa.

\subsection{Client Architecture}
\paragraph{Model View ViewModel}
The architecture of the front end user interface is based around the Model View ViewModel (MVVM) design pattern. Facilitated by the JS library Knockout.js, MVVM allows us to consolidate the logic associated with the DOM (Domain Object Model) into one place \cite{knockout}.
We no longer need to apply JQuery or Vanilla Javascript to individual DOM elements through classes and ids, but instead we can apply rules identified in the ViewModel to various elements throughout the DOM. This allows for dependency tracking of JS variables and automatic real time updates as their values change.
This lends itself well to the problem domain of many rapidly changing documents and results sets as the user identifies relevant documents. \\
Knockout was chosen due to developer familiarity. There are various other libraries that allow for similar designs, however familiarity with Knockout's syntax and best practices meant that development was easy and fast.
This was important as the learning curve for the backend technologies required more attention, and so ease of development on the frontend was key.

\paragraph{Style}
Bootstrap was chosen mainly due to familiarity reasons. It provides myriad customisable components, as well as being extremely well documented \cite{bootstrap}. This dramatically simplifies the rapid development of a user interface.

\paragraph{AJAX}
JQuery is used in the frontend to interface with the backend through HTTP (AJAX) requests. JQuery provides a very well documented and widely adopted API for this. JQuery is a dependency of both Knockout and Bootstrap and so utilizing it's already present features was sensible and easy.

\paragraph{Web Application Framework}
The web application currently runs on top of an Express.js server running on Node.js \cite{express} \cite{node}. This is not particularly necessary as the web application could, with some refactoring, be served through a static web server such as GitHub Pages. What Express.js does allow for, however, is the use of the EJS template system. EJS allows us to define partial HTML pages and insert them into others. This is useful within the MVVM design as we can initialize compartmentalized parts of the DOM with specific ViewModels using the \code{with} binding, see the Client Implementation (\ref{viewmodels}) chapter for more details.

%%%%%%%%%%%%%%%%%%%%%%%%%%%%%%%%%%%%%%%%%%%%%%%
\chapter{Server Implementation}
\section{Tools}
\paragraph{Git and GitHub}
Git was used, alongside GitHub as Source Control Management (SCM) for the project \cite{git} \cite{github}. This allows one to maintain version history for the project. Hosting the repository on GitHub also served as a back-up for the project, and allows it to be cloned onto any machine.
\section{Dependency Management and Build System}
Maven was used as a dependency management and build system \cite{maven}. Maven was chosen due to it's ease of use and clear documentation. It allowed for easy inclusion of all the dependencies needed for the project such as Terrier, StanfordNLP and Jersey. Maven's build system also ensures all unit and integration tests pass before building the system into a runnable jar file.

\section{Data}
\subsection{Source Documents}
During development the collection of source documents used for testing and experimentation was based upon the United States Diplomatic Cables Leak, as released on Wikileaks. This collection has been confirmed to be relevant and very similar to the type of documents reviewed by the archives services in the UK \textbf{do you have a source for this?}. The documents were provided in TREC format, with one document per file. They were arranged into directories based on year and month of creation.
\begin{minted}{xml}
<DOC>
<DOCNO>08ABUDHABI1104</DOCNO>
<CREATED>2008-09-28</CREATED>
<RELEASED>2011-08-26</RELEASED>
<CLASSIFICATION>UNCLASSIFIED//FOR OFFICIAL USE ONLY</CLASSIFICATION>
<ORIGIN>Embassy Abu Dhabi</ORIGIN>
<FROM>AMEMBASSY ABU DHABI</FROM>
<TO>RUEHC/SECSTATE WASHDC 1507</TO>
<SUBJECT>UAE PRESIDENT DEPLOYS PERSONAL CHARITY FOUNDATION </SUBJECT>
UAE PRESIDENT DEPLOYS ...
</DOC>
\end{minted}
\subsection{Target Documents}
A combination of Associated Press and Reuters articles were used as target documents to perform retrieval on. These were again in TREC format, with one file per document. They were also each compressed due to the extremely large number of them.
They were given in the following format:
\begin{minted}{xml}
<DOC>
<DOCNO>trc2_168664420090204</DOCNO>
<TITLE>GLOBAL MARKETS-Economy hopes boost stocks, dollar</TITLE>
<DATE>2009-02-04</DATE>
<KEYWORDS> MARKETS GLOBAL   </KEYWORDS>
* Global stocks rise on economy hopes...
</DOC>
\end{minted}
\section{Indexing}
\subsection{Target Documents}
The system operates on the assumption that there already exists a target index within it's directory structure.
As such, a separate Java program was written which can index target documents through command line invocation. The program loads the appropriate Terrier properties and invokes indexing through Terrier's API. During target document indexing we set the Terrier properties so as to create abstracts. Abstracts are parts of a tagged document to be saved in meta-data to provide helpful snippets of the document when it is found during retrieval. In this case, we generate abstracts for the title, data, keywords and body of the document to give an idea of document contents when looking at query results.
The correct configuration of Terrier to achieve this caused considerate trouble during development and was actually the identifier of a flaw in Terrier itself. Owing to the fact that the project supervisor was a key Terrier developer, this was fixed through a patch and the target index could be generated.
As target document indexing uses the same tokeniser (see \ref{nertok})  as source document indexing, we can guarantee term matching across the two corpora when performing retrieval.

\subsection{Source Documents}
Source indexing is invoked through the REST API. Again this loads the relevant Terrier properties and invokes indexing through the Terrier API. The Terrier properties important to this operation are the \code{TrecDocTags.PropertyTags}, which identify the tags in a tagged document which are to be saved as document properties, rather than being indexed. Additionally, source indexing also invokes query generation (see \ref{querygen}). 

\paragraph{Available Indexes}
Since we can index individual directories of source documents, a tool to choose which collection to review must also be provided. This functionality relies on a system of recursive file finding. The psuedocode is as follows:

\subsection{Query Generation} \label{querygen}
Immediately after indexing a collection of source documents, the system invokes the query generation process. We create 4 queries: All Terms, Named Entity Terms, Top Ten Tf-Idf Named Entities and Subject Queries.
To create the All Terms and Named Entity Queries we simply iterate through the terms in a given document and append them to a \code{StringBuilder}. One \code{StringBuilder} amalgamates every term found, while the other only takes terms containing an underscore, the identifier for a named entity found during indexing using the \code{NamedEntityTokeniser}.
For the Tf-Idf query we leverage the Terrier API which has methods to give us all the data we need.
Tf-Idf (Term Frequency - Inverse Document Frequency) is defined as:
\begin{gather*}
Tf\textnormal{-}Idf = Tf \cdot Idf \\
\textnormal{where:} \\ 
Tf = \frac{Number\ of\ Times\ Terms\ Appears\ in\ Document}{Total\ Number\ of\ Terms\ in\ Document} \\ \\
Idf = \ln{\frac{Total\ Number\ of\ Documents\ in\ Collection}{Number\ of\ Documents\ Containing\ Relevant\ Term}}
\end{gather*}
For any given term we can retrieve it's term and document frequency in order to easily calculate the Tf-Idf of the term. Once calculated the term and value are inserted into a data structure from which retrieve the top-ten terms to formulate the query.\\
Since the subject is not indexed like the body of the document we simply retreive the subject from the meta index. We then remove all punctuation and run it through the Named Entity Analysis to obtain a query which will match against the target index. \\
Having formulated all four queries we wrap them in tags and append them to the end of the trec source document from which they were generated. This allows the process described in \ref{docparse} to find the queries without the need for any new storage mechanisms.

\subsection{Named Entity Tokeniser} \label{nertok}
We wanted to identify Named Entities within documents to enhance query generation and provide better retrieval. Terrier indexing uses a tokeniser to identify terms to add to the index. This provided a hook-in point for named entity identification as we can identify if terms are Named Entities before they are passed out of the tokeniser to be added into the index.
A custom subclass of the Tokeniser class in Terrier was implemented which applied named entity identification to a document body before being tokenised.
If a term was identified to be a named entity by Stanford NLP then an underscore and the named entity type was appended to the terms. For example, 
\code{kelvin} becomes \code{kelvin\textunderscore person}.
This tokeniser was used for both source and target indexing to ensure matching during retrieval.
The logic for annotating a section of text was further abstracted to a seperate class so that it could be reused to annotate any text, outside of the context of tokenisation during indexing. This was used specifically to annotate subject queries in the query generation phase, but could provide further applications in the future.
\paragraph{Classifiers and Distributional Similarity}
An instance of \code{StanfordCoreNLP} must be created to use the named entity annotation features.
In this instance the fewest possible annotators were used to acheive named entity recognition.
We use the simplest classifier model provide by StanfordNLP which only looks for Location, Organisation and Person named entities.
As discussed in the evaluation section, we also tried Named Entity models with and without Distributional Similarity Features (The documentation suggests this can improve performance while sacrificing efficiency).
We also avoid time and numeric entity identification, so as to only focus on the tangible entities mentioned throughout each document.

\section{Retrieval}
Retrieval is a fairly simple affair, almost identical to the provided example retrieval code provided in the Terrier documentation.
\paragraph{Decoration}
At this stage we also leverage Decoration, a functionality provided by Terrier to highlight and emphasise terms within a result set which appear in the query. Conveniently the decoration wraps found terms in \code{<b>} HTML tags, allowing us to render them as bold in the frontend.

\section{Document Parsing} \label{docparse}
In order to display the source and target documents to users on the front end they must be parsed and converted to JSON.
Both target and source documents are provided in TREC format with various meta-data tags and an untagged body section.
An abstract class, \code{TaggedDocument} was created which reads these documents identifying tag types and content. Once identified, this method can pass the tag type and tag content to the concrete subclass which implements a method

\section{Evaluation Preparation}
In order to prepare the \code{.topics} and \code{.qrels} files necessary for offline evaluation. 

\section{Properties}
Terrier operates with a collection of properties which change the behaviour of it's various operations. These properties can be set inside a terrier.properties file, however this complicates changing these properties between different operations. These properties can be set programatically and so a static class was made which allows for easy switching of settings to facilitate correct invokation of indexing and retrieval. : Talk about Source vs. Target indexing

\section{Test Collection Generation}
To allow for the offline evaluation identified in the : Insert section API methods were added to allow addition of topics and qrels.
Reading/Writing to these files.

\section{Summary}
research questions for eval chapter

%%%%%%%%%%%%%%%%%%%%%%%%%%%%%%%
\chapter{Client Implementation}
This chapter will deal with the steps taken in the implementation of the web application which acts as the client in the client server paradigm.

\subsection{Tools and Methods}
\paragraph{Dependency Management}
Bower was used for dependency management within the realm of frontend JavaScript. Bower allows a programmer to define dependencies through a \code{bower.json} file. These can then be downloaded through a bower install command straight into a bower components folder. This was used to install JQuery, Bootstrap and Knockout.

\paragraph{IDE}
JetBrains WebStorm was used to develop the client web application. It allows for easier and faster development by allowing one to run a Node.js application through it's interface, as well as other niceties such as intelligent code completion.

\section{ViewModel and Data Binding} \label{viewmodels}

\section{Computed Functions and Observables}
Within the ViewModel, the programmer defines Observables instead of plain JS variables. These ensure inclusion of the logic needed to interact with the DOM through data-binding. Observables can also be ``computed". This means that their value depends on an automatically evaluating function. These functions re-evaluate any time an observable which is a dependency to the function changes. This system allows for complex logic which can be fired off automatically depending on certain circumstances.
This feature can also be exploited to allow for automatic firing of events not related to any specific observable. This is very useful in this project's context as we can send off API requests when certain observables change, for example the requested source document.

\section{Components for Tab System}
Knockout provides a component system which allows one to register new html elements, along with a specific view model with which these objects interact. The programmer provides a html template (complete with data-binding) and a ViewModel. When instantiating one of these components one passes in arguments to fill the ViewModel.
This system lends itself excellently to the tab system present in the system. Using a ``foreach" binding along with these custom components a fully functional tab system exists, which is, at it's foundation, simply an array of docNo's. Such is the power of declarative bindings and custom components! Haha!

\section{Summary}
In order to validate user acceptance testing it was decided that variants of the front end system were to be created. These variants would play on some of the main features of the frontend design, in order to identify the UI elements which were helping and hindering the user experience.

%%%%%%%%%%%%%%%%%%%%%%%%%%%%%%%%%%%%%%%%%%
\chapter{Evaluation}
\section{Testing}
\subsection{Unit Testing}
JUnit was used to organise and run unit tests on the various classes present in the backend application.
The Arrange-Act-Assert pattern was used to organise unit tests in a structured way. This arrangement allows for clear identification of the method being tested, separate from the code needed to prepare for the test.
Testing necessitated the refactoring of some of the files which read and write information to files. Let us observe an example in the form of the TargetDocument.java class. This class analyses a compressed target document to create a representation which can be converted to JSON and delivered to the Web App for viewing.
The constructor originally took a file path and constructed the necessary streams and readers to allow for reading of the file.
\begin{minted}{java}
public TargetDocument(String path){
  try {
    FileInputStream stream = new FileInputStream(path);
    GZIPInputStream gzStream = new  GZIPInputStream(stream);
    InputStreamReader inputStreamReader = new  InputStreamReader(gzStream);
    BufferedReader br = new  BufferedReader(inputStreamReader);
    documentParser(br);
...
\end{minted}

This approach is inflexible as it assumes the file path will direct to a gzipped file. It means that unit tests need either a sandbox file system to prepare tests, or some kind of mocking of the file system. Mocking a file system is overly complex and adding a sandbox filesystem adds unpredictable side effects.
A better, more general, and self contained approach is to have the constructor take a \code{Reader} interface as an argument. We can then instantiate the necessary \code{BufferedReader} needed.

\begin{minted}{java}
public TargetDocument(Reader r){
    BufferedReader br = new BufferedReader(r);
    this.docNo = this.title = this.date = this.keywords = this.body = "";
    documentParser(br);
...
\end{minted}

This approach was extended to several classes which handle interacting with files.
Code which interacted with the Terrier API posed problems in term of unit testing. Terrier often introduces side effects in the form of interacting with established index files in the file system.
\section{Qualitative Analysis and Defining Relevance}
Through the course of developing the system and producing the ground truth necessary for the Offline Evaluation as discussed in the next section, certain nuances of the system became clear.
It was clear that named entity analysis was certainly not always full proof. Sometimes entities were missed and often words such as ``and'' or ``the'' were tagged as named entities incorrectly. Combating this, the variations of ``and'' and ``the'' created in the named entity identification process were added to the stopwords file to avoid them appearing in queries and being matched in retrieval.
In order to produce the \code{qrels} file it was necessary to define returned target documents as ``relevant'' or ``not relevant''. This was extremely nuanced and relied on the discretion of the reviewer.
Sometimes dates were important - example Fighting in North Darfur.
Othertimes they were not. Cluster Munitions Japan for example.
A single word can throw off the search results. ``Congress''.
The query formed from Tf-Idf ranking of named entities was interesting. It often produced relevant results, but was also the cause of many strange results, such as reports on international sports fixtures. One can see how this occurred, since the names of several countries (often small African nations) not mentioned frequently throughout the collection can push these terms to the top of this query.

\section{Offline Evaluation Using Terrier}
\begin{filecontents*}{data.csv}
name map length time
allterms 0.4554  358.5  2.51435
ne 0.3979  93.45  0.4132
tfidf 0.3038  10  0.13675
subject 0.1593  9.75  0.1688
\end{filecontents*}
\subsection{Aims and Method}
The aim of this quantitative evaluation was to gather data relating to the performance of the variations of queries formed from source documents. These results could then be used to determine which query was most appropriate to run automatically upon source document loading. \\
The methods used to produce these queries is discussed at length in the Server Implementation chapter, see \ref{querygen}.
Offline evaluation was performed using the evaluation tools provided with Terrier. This evaluation allows one to produce detailed statistics regarding the performance of an IR system.\\
For 20 source documents, a number of target documents were reviewed and rated as either relevant or non relevant. This formed the ground truth against which the evaluation could be performed.\\
We were most interested in the Mean Average Precision scores generated by the offline evaluation.\\
Mean Average Precision (MAP) is a measure of the effectiveness of an IR system.
It is given by:
\begin{displaymath}
  MAP=\frac{\sum_{n=1}^{Q} Ave(P(q))}{Q}
\end{displaymath}
Where: 
\begin{itemize}
\item{~$Q$ is the number of queries.}
\item{~$Ave(P(q))$ is is the average precision of a given query.}
\end{itemize}

Also considered was the effects of Document At A Time (DAAT) vs. Term At A Time (TAAT) retrieval. This means whether a full document is considered using all the terms from a query (DAAT) or each term is considered concurrently for all documents (TAAT). In order to ensure the choice of query conformed to the non-functional requirements, specifically \textbf{NF.3}, we also consider mean retrieval time.
\subsection{Results}
\paragraph{Distributional Similarity in Named Entity Identification}
\paragraph{Precision At 1 and 5}
The query which performed the worst was the query formed from the subject line. This can be expected, as other than running Named Entity recognition on the line, no other analysis was done to improve it's performance. There is also no guarantee that the subject line will contain keywords which provide vital context to a query.\\
Take, for example, this subject from a document in the source collection:
\begin{center}``PARLIAMENT'S FALL SESSION: A PREVIEW"\end{center}
We are given no insight into which country this refers to and although this is not true for many documents, the results clearly demonstrate the lack of depth in subject queries causes under-performance.\\
The Tf-Idf Named Entities query scores almost double the subject query. It is limited to the top 10 highest scoring named entities in terms of Tf-Idf, meaning the query is highly relevant to the given document. Here terms are weighted more heavily if they appear frequently in the source document, but infrequently in the source collection. This (in theory) generates a more specific, focused query. \\
The query consisting of all named entities identified in the document was second only to the query of all terms. This suggests that the inclusion of verbs and other non named entities produces a more meaningful query.\\
We can see from Table \ref{results} and Figures \ref{fig: timegraph} and \ref{fig: lengthgraph}, that the most highly performing queries consist of more terms and take longer to process.\\
Although the All Terms Query produced the highest MAP scores, it took upwards of 5x as long to run compared to the Named Entities query. At this point, one must weigh up the importance of relative result relevance against time concerns. Users are only willing to wait so long for results TODO: Reference. As such, the most appropriate query to run automatically is that which consists of Named Entities.

\begin{center}
\begin{table}[h]
\centering
\begin{tabular}{|c|c|c|c|}
\hline
Query                 & MAP    & Mean Query Length & Mean Query Processing \\ 
& & (Words) & Time (s) \\\hline
All Terms             & 0.4554 & 358.5             & 2.51435                        \\ \hline
Named Entities        & 0.3979 & 93.45             & 0.4132                         \\ \hline
Tf-Idf Named Entities & 0.3038 & 10                & 0.13675                       \\ \hline
Subject               & 0.1593 & 9.75              & 0.1688                        \\ \hline
\end{tabular}
\caption{Terrier Evaluation Results}
\label{results}
\end{table}
\end{center}

\begin{figure}[h]
\begin{tikzpicture}
 \begin{axis}[
 	ylabel=$MAP$,
    xlabel={Mean Query Processing Time},
    width=0.9\textwidth,
    height=0.4\textwidth
    ]
        \addplot table[x=time,y=map] {data.csv};
    \end{axis}
\end{tikzpicture}
\caption{Mean Query Processing Time vs. MAP} \label{fig: timegraph}
\end{figure}
\bigskip
\begin{figure}[h!]
\begin{tikzpicture}
 \begin{axis}[
 	ylabel=$MAP$,
    xlabel={Mean Query Length},
    width=0.9\textwidth,
    height=0.4\textwidth
    ]
        \addplot table[x=length,y=map] {data.csv};
    \end{axis}
\end{tikzpicture}
\caption{Mean Query Length vs. MAP} \label{fig: lengthgraph}
\end{figure}
\subsection{Conclusion}
Continue running with indexes built upon no distributional similarity.
Auto run the named entities query.
Is it unreasonable to use both the distim and no-distim indexes depending upon the query being run? - It provides a massive improvement to the tf-idf named entities and subject queries.

\section{User Evaluation}
\subsection{Think-Aloud}

Timothy Gollins, a key potential beneficiary of such a system participated in a small think-aloud study while examining the software in use. As former Head of Digital Preservation at The National Archives (UK), and current Head of Digital Archiving at the National Records of Scotland, Timothy has expert insight into the problem domain.
Some of the key points learned from this session are listed below:

\begin{itemize}
\item \textit{``Named Entity importance is nuanced and frequency (or Tf-Idf) of named entities does not necessarily produce the most important terms.''}
\par
This raises an important point regarding the limitations of automatic query generation.\textbf{There is of course the ability to run custom queries. But this could be improved somewhat}

\item \textit{``Further term highlighting or importance weighting is necessary to help produce the best results.''}
\par
This point highlight the fact that the queries generated right now are fairly general, and with some deep understanding of the way archivists review documents could help produce better results. Briefly mentioned was weighting the first and last paragraphs higher than the middle of the document, as these are likely to contain the most pertinent information.

\item \textit{``More advanced methods could be leveraged to improve yet further the capabilities of the software, like eye tracking or automatic querying upon highlighting''}
\par
Although beyond the scope of this project, this point identifies the potential expansion of this project. Eye-tracking could be used to automatically track what the user is interested in from the source document, forming the basis of new automatic queries. A simplified version of this could use text highlighting to accomplish the same results.

\item \textit{``UI considerations are equally as important as information retrieval considerations when it comes to improving the effectiveness of the software''}
\par
At this point in the project, the implementation of most of the IR functionality was complete. It now became important to consider how best to modify and improve the user interface to ensure the easiest experience possible. - DECORATION
\end{itemize}

\textbf{Talk about how this allowed iteration and improvement}

\section{Refining the System}
What did we do to make the system better:
Chose the best query.
Improved the stopwords file slightly.

What could we do to make the system better
Add yet more stopwords
Consider dates.

\section{Conclusion}

%%%%%%%%%%%%%%%%%%%%%%%%%%%%%%%%%%%%%%%%
\chapter{Conclusions}
\section{Fulfilled Requirements}
The Identifying Public Domain Knowledge project set out to create a system capable of improving the document review system necessary as digital records become more and more widespread.
The final product satisfies all of the Must Have requirements and addresses some of the Should and Could Have sections. 
\section{Possible Improvements}
Machine Learning. Eye Tracking. More Query formulations to analyse.

\section{Reflection}

%%%%%%%%%%%%%%%%
%              %
%  APPENDICES  %
%              %
%%%%%%%%%%%%%%%%
\begin{appendices}  r

\end{appendices}

%%%%%%%%%%%%%%%%%%%%
%   BIBLIOGRAPHY   %
%%%%%%%%%%%%%%%%%%%%

\bibliographystyle{plain}
\bibliography{bib}

\end{document}
